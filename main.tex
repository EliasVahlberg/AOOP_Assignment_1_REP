\documentclass{article}
\usepackage[utf8]{inputenc}
\usepackage{listings}
\usepackage{xcolor}

\definecolor{codegreen}{rgb}{0,0.6,0}
\definecolor{codegray}{rgb}{0.5,0.5,0.5}
\definecolor{codepurple}{rgb}{0.58,0,0.82}
\definecolor{backcolour}{rgb}{0.95,0.95,0.92}

\lstdefinestyle{mystyle}{
    backgroundcolor=\color{backcolour},   
    commentstyle=\color{codegreen},
    keywordstyle=\color{magenta},
    numberstyle=\tiny\color{codegray},
    stringstyle=\color{codepurple},
    basicstyle=\ttfamily\footnotesize,
    breakatwhitespace=false,         
    breaklines=true,                 
    captionpos=b,                    
    keepspaces=true,                 
    numbers=left,                    
    numbersep=5pt,                  
    showspaces=false,                
    showstringspaces=false,
    showtabs=false,                  
    tabsize=2
}

\lstset{style=mystyle}


\title{AOOP Assignment 1 }
\author{elias.vahlberg.2 }
\date{April 2021}

\begin{document}

\maketitle

\section*{Overview}
\section*{Exercises}
\subsection*{1) (*) Exercise 2.9 in the course book.}
\textit{What relationship is appropriate between the following classes: aggregation,
inheritance, or neither?}\\
\indent \textbf{\textcolor{blue}{Solution:}} \\
\textbf{a) Student–Course}\\
\indent (n,n) \\
\indent    One student can have multiple courses.\\
\indent    One course can have multiple students.\\
\textbf{(b) Course–Section}\\
\indent    (1,n)\\
\indent    One section can only belong to one course.\\
\indent    One course can have multiple sections.\\
\textbf{(c) Section–Instructor}\\
\indent    (n,n)\\
\indent    One section can have multiple instructors.\\
\indent    One instructor can have multiple sections.\\
\textbf{(d) Section–Room}\\
\indent    (n,n)\\
\indent    One section can use multiple rooms.\\
\indent    One room can be used by multiple sections.\\
\subsection*{2) Exercise 2.12 in the course book.}
\textit{Consider an online store that enables customers to order items from a catalog and
pay for them with a credit card. Draw a UML diagram that shows the relationships
between these classes:}\\
\indent \textbf{\textcolor{blue}{Solution:}} \\
\subsection*{3) Exercise 2.13 in the course book.}
\textit{Consider this test program:}
\lstinputlisting[language=Java]{Tester.java}
\textit{Draw a sequence diagram that shows the method calls of the main method.} \\
\indent \textbf{\textcolor{blue}{Solution:}} \\
\textbf{***IMAGE}


\end{document}
