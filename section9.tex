\documentclass[main.tex]{subfiles}
\begin{document}
\subsection*{9) (*) Re-implement the stack class from assignment 7 with message elements.}
\textit{Re-implement the stack class from assignment 7 this time with message elements (same as
elements in message queue) and implement a method size() which returns the number of the
messages in the stack.}\\
\indent \textbf{\textcolor{blue}{Solution:}} \\
MessageStack.java : \ref{lst:MessageStack} \\


\subsection*{10) Add the methods multAdd() and multRemove() to the message queue class}
\textit{Add two following methods to the message queue class (which is a circular array). The first
method, multAdd(m1, m2, ..., mn), is an extension of the add method which adds a set of
elements to the circular array instead of one element. The second method, multRemove(n), is an
extension of the remove method which removes n elements from the queue instead of one. Check
the correctness of your implementation using JUnit.}\\
\indent \textbf{\textcolor{blue}{Solution:}} \\
MessageQueue.java : \ref{lst:MessageQueue} \\
MessageQueueTest.java : \ref{lst:MessageQueueTest}


\subsection*{12) (*) Exercise 4.1 in the course book}
\textit{When sorting a collection of objects that implements the
Comparable type, the sorting method compares and rearranges the objects. Explain the role of
polymorphism in this situation.}
\indent \textbf{\textcolor{blue}{Awnser:}} \\
    The comparable interface allows the use of sorting based on the inherited method compareTo().
    This is internally unique but from an external point of view it handled the same.
    The word Polymorphism means "many forms" which exactly describes the function compareTo()
    since it is used to preform many different tasks depending on where it is implemented.


\subsection*{13) Exercise 4.6 in the course book}
\textit{Write a method
public static String maximum(ArrayList<String> a, Comparator<String> c) { . . . }
that computes the largest string in the array list using the ordering relation that is defined by the
given comparator. Supply a test program that uses this method to find the longest string in the
list.}\\
\indent \textbf{\textcolor{blue}{Solution:}} \\
Exercise406.java : \ref{lst:Exercise406}

\subsection*{14) (*) Exercise 4.9 in the course book}
\textit{Define an interface type Filter as follows:
}\\
Filter.java : \ref{lst:Filter}
Then suply a method:
\begin{verbatim}
    public static String[] filter(String[] a, Filter f) { . . . }   
\end{verbatim}
\textit{that returns an array containing all the elements of a that are accepted by the filter. Test your
method by filtering an array of strings and accepting all strings that contain at most three characters.
}\\
\indent \textbf{\textcolor{blue}{Solution:}} \\
Exercise409.java : \ref{lst:Exercise409}

\subsection*{15) (*) Exercise 4.14 in the course book}
\indent \textbf{\textcolor{blue}{Solution:}} \\
Exercise414.java : \ref{lst:Exercise414}\\
CircleIcon.java : \ref{lst:CircleIcon}\\


\end{document}